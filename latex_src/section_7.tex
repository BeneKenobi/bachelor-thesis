\section{Conclusion}\label{sec:conclusion}

In conclusion, this bachelor thesis provides a comprehensive insight on the significance of implementing a \acl{SWfMS} to support the transition to a newer reference genome, in this case \textit{GRCh38}. By leveraging the included resource usage reports of the chosen solution \textit{Nextflow}, it is possible to improve the efficiency of the genetic analysis pipeline greatly, with a reduction in resource usage by over a third. 

The adoption of \textit{Nextflow} represents the beginning of a journey for the \ac{DoHG@MHH} to use their genetic analyzation pipeline more professional. This work lays the foundation for further advancements, such as the separation of \textit{\ac{megSAP}} steps and the addition of \textit{Nextflow Tower}. These suggestions, if implemented, would provide the \ac{DoHG@MHH} with even greater benefits in terms of efficiency, accessibility, and usability.

One of the main challenges that the \ac{DoHG@MHH} faces is limited processing capacity, which has been remedied by the implementation of \textit{Nextflow}. In addition, the cost of using cloud infrastructure has also been presented as a possible solution in the future. The use of cloud infrastructure would provide the \ac{DoHG@MHH} with unlimited processing capacity, and therefore, the ability to perform genetic analyses with as much parallelism as needed. Especially for the task at hand, to reanalyze all old samples with a new reference genome without disturbing regular diagnostics, the cost of such an endeavor might be worth it.

The results of this thesis demonstrate that the implementation of a \ac{SWfMS} into the genetic analysis pipeline has many benefits, not only in terms of efficiency and accuracy, but also in terms of future scalability. This transition to \textit{GRCh38} has been made possible due to the use of \textit{Nextflow}, and the results will be transferred to the routine medical diagnostics pipeline, as \textit{Nextflow} will be adopted there. The migration will be seamless, with the current scripts being replaced by a script that starts the new \textit{Nextflow} workflow.

Finally, this thesis provides a clear and concise overview of the implementation of a \ac{SWfMS} to support the transition to a new reference genome. The results demonstrate that the use of \textit{Nextflow} is an effective solution, not only for the \ac{DoHG@MHH}, but also for any laboratory that needs to transition to a new reference genome. The benefits of using a \ac{SWfMS}, such as \textit{Nextflow}, are numerous and include improved efficiency, scalability, and cost savings. This thesis serves as a blueprint for the successful implementation of a \ac{SWfMS} in a genetic analysis pipeline and provides a roadmap for future improvements and advancements.